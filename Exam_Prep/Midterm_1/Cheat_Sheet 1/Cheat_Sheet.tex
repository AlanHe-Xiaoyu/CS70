\documentclass{article}
\usepackage[left=3cm, right=3cm, top=3cm]{geometry}
\usepackage{amssymb}
\begin{document}

\noindent{\Large Quantifiers}
\begin{itemize}
\item $\forall x\forall y, P(x,y)\equiv\forall y\forall x, P(x,y)$ 
\item $\exists x\exists y, P(x,y)\equiv\exists y\exists x, P(x,y)$ 
\item $\forall x\exists y, P(x,y)\not\equiv\exists y\forall x, P(x,y)$
\begin{itemize}
	\item $\forall x\exists y, P(x,y)\not\Longrightarrow \exists y\forall x, P(x,y)$
	\item $\exists y\forall x, P(x,y)\Longrightarrow \forall x\exists y, P(x,y)$
\end{itemize}
\item $\forall x \big(P(x)\land Q(x)\big)\equiv\big(\forall x, P(x)\big)\land\big(\forall x, Q(x)\big)$
\item $\forall x \big(P(x)\lor Q(x)\big)\not\equiv\big(\forall x, P(x)\big)\lor\big(\forall x, Q(x)\big)$ \\
Let $P(1) = Q(2) = True, P(2) = Q(1) = False,$ then LHS $= True$, RHS $= False.$
\item $\exists x \big(P(x)\land Q(x)\big)\not\equiv\big(\exists x, P(x)\big)\land\big(\exists x, Q(x)\big)$ \\
Let $P(1) = Q(2) = True,$ all other cases $= False,$ then LHS $= False$, RHS $= True.$
\item $\exists x \big(P(x)\lor Q(x)\big)\equiv\big(\exists x, P(x)\big)\lor\big(\exists x, Q(x)\big)$ \\
\end{itemize}

\noindent{\large Note 2}\\
(Direct Proof, Proof by Contraposition, Proof by Contradiction, Proof by Cases)
\begin{itemize}
\item Theorem 2.1: For any $a, b, c\in\mathbb{Z}$ if $a\mid b$ and $a\mid c,$ then $a\mid(b + c)$.
\item Theorem 2.2. Let $0 < n < 1000$ be an integer. If the sum of the digits of $n$ is divisible by 9, then $n$ is divisible by 9.
\item Theorem 2.3 (Converse of Theorem 2.2). Let $0 < n < 1000$ be an integer. If $n$ is divisible by 9, then the sum of the digits of $n$ is divisible by 9.
\item Theorem 2.4. Let $n\in\mathbb{Z^+}$ and let $d\mid n$. If n is odd then d is odd.
\item Theorem 2.5 (Pigeonhole Principle). Let $n, k\in\mathbb{Z^+}$ be positive integers. Place $n$ objects into $k$ boxes. If $n > k$, then at least one box must contain more than one object.
\item Theorem 2.6. There are infinitely many prime numbers.
\item Lemma 2.1. Every natural number greater than one is either prime or has a prime divisor.
\item Theorem 2.7. $\sqrt{2}$ is irrational.
\item Lemma 2.2. If $a^2$ is even, then $a$ is even.
\item Theorem 2.8. There exist irrational numbers x and y such that $x^y$ is rational. \\
\end{itemize}

\noindent{\large Note 3}\\
(Base Case, Inductive Hypothesis, Inductive Step)
\begin{itemize}
\item Theorem 3.1: $\forall n\in\mathbb{N}, \sum\limits_{i=0}^{n} i = \frac{n(n+1)}{2}$
\item Theorem 3.2: $(\forall n\in\mathbb{N})\ \big(3\mid(n^3-n)\big)$
\item Theorem 3.3: Let $P(n)$ denote the statement ``Any map with $n$ lines is two-colorable''. Then, it holds that $(\forall n\in\mathbb{N})\ \big(P(n)\big)$
\item Theorem 3.4: $\forall n\geq1$, the sum of the first $n$ odd numbers is $n^2.$
\item Theorem 3.5: $(\forall n\geq1)\ \big(\sum\limits_{i=1}^{n} \frac{1}{i^2}\leq(2 - \frac{1}{n})\big)$
\item Theorem 3.6: For every natural number $n\geq12$, it holds that $n = 4x + 5y$ for some $x, y\in\mathbb{N}$.
\item Theorem 3.7: Every natural number $n>1$ can be written as a product of primes.
\end{itemize}

\noindent {\large Note 4}\\
(Stable Marriage Algorithm)
\begin{itemize}
	\item Lemma 4.1: The stable marriage algorithm always halts.
	\item Lemma 4.2 (Improvement Lemma): If man $M$ proposes to woman $W$ on the $k^{th}$ day, then on every subsequent day $W$ has someone on a string whom she likes at least as much as $M$.
	\item Definition 4.1 (Well-ordering principle): If $S\subseteq N$ and $S\neq\emptyset$, then $S$ has a smallest element.
	\item Lemma 4.3: The stable marriage algorithm always terminates with a pairing.
	\item Theorem 4.1: The pairing produced by the algorithm is always stable.
	\item Definition 4.2 (Optimal woman for a man): For a given man $M$, the optimal woman for $M$ is the highest woman on $M$'s preference list that $M$ is paired with in any stable pairing.
	\item Theorem 4.2: The pairing output by the Stable Marriage algorithm (with relaxed proposal as well) is male optimal, and (Proved in HW:) no two men can have the same optimal partner.
	\item Theorem 4.3: If a pairing is male optimal, then it is also female pessimal.
	\item There always exists some woman who receives a single proposal, and she always receives it on the last day of the algorithm. (Proved both in discussion and homework)
	\item Extra (need proof by induction on exam): The most rejections that can occur on $k^{th}$ day is $n - k.$
	\item Extra (good as counterexample): Consider rotational preferences, Man $A: 1>2>3; B:2>3>1; C:3>1>2$ and Woman $1: B>C>A; 2: C>A>B; 3: A>B>C$, which has 3 stable pairings.
\end{itemize}

\noindent {\large Note 5} \\
(Planar Graph, Tree, Complete Graph, Hypercube)
\begin{itemize}
\item path - simple (no repeating vertex)
\item cycle - closed path
\item walk - path w/ possibly repeating vertex
\item tour - closed walk with NO repeating edge
\item Eulerian - every edge once
\item Hamiltonian - every vertex once
\item Theorem 5.1 (Euler’s Theorem): An undirected graph $G$ = $(V,E)$ has an Eulerian tour $\iff G$ is even degree, and connected (except possibly for isolated vertices).
\end{itemize}
Function $EULER(G,s)$: \\
\indent $T = FINDTOUR(G,s);$ \\
\indent Let $G_1,...,G_k$ be the connected components when the edges in $T$ are removed from $G$, and let $s_i$ be the first vertex in $T$ that intersects $G_i$; \\
\indent Output $SPLICE(T,EULER(G_1,s_1),...,EULER(G_k,s_k))$ \\
end $EULER$
\begin{itemize}
	\item Theorem 5.2 (Euler’s formula): For every connected planar graph, $v + f = e + 2\ (\equiv e\leq 3v - 6)$
	\item Theorem 5.3: A graph is non-planar $\iff$ it contains $K_5$ or $K_{3,3}$.
	\item (Prove tree properties with induction.) $G = (V, E)$ is a Tree $\iff$:
	\begin{enumerate}
		\item $G$ is connected and contains no cycles.
		\item $G$ is connected and has $n-1$ edges (where $n = |V|$ is the number of vertices).
		\item $G$ is connected, and the removal of any single edge disconnects $G$.
		\item $G$ has no cycles, and the addition of any single edge creates a cycle.
	\end{enumerate}
	\item Theorem 5.4: The statements ``$G$ is connected and contains no cycles'' and ``$G$ is connected and has $n-1$ edges'' are equivalent.
	\item Lemma 5.1: The total number of edges in an $n$-dimensional hypercube is $n2^{n-1}.$
	\item Theorem 5.5: Let $S\subseteq V$ be such that $|S|\subseteq|V-S|$ (i.e., that $|S|\leq2^{n-1}$), and let $E_S$ denote the set of edges connecting $S$ to $V-S$, (i.e., $E_S := \big\{\{u, v\}\in E \mid (u\in S) \land (v\in V-S) \big\}$). Then, it holds that $|E_S|\geq|S|$.
	\item $\sum\limits_{v\in V} deg(v) = 2|E|$ (sum of all degrees = twice the number of edges) (Proved earlier in class)
	\item Extra (Proved in HW): A graph is bipartite $\iff$ it has no tours of odd length, and $\forall n\in\mathbb{Z^+},$ the $n$-dimensional hypercube is bipartite (so a hypercube has no tours of odd length and can be two-colored).
	\item Extra (need proof): A graph with $k$ edges has $\geq|V| - k$ connected components.
	\item Extra (proof by PHP): In any (simple) graph, there are always two vertices of the same degree.
	\item Extra (need proof): $K_n$ can be vertex colored with $n$ colors.
\end{itemize}

\noindent {\large Note 6, 7} \\
(Modular and FLT)
\begin{itemize}
	\item Theorem 6.1: If $a\equiv c$ and $b\equiv d\pmod{m}$, then $(a+b)\equiv(c+d)$ and $(ab)\equiv(cd)\pmod{m}$
	\item Theorem 6.2: Let $m,x$ be positive integers such that gcd$(m,x) = 1$. Then $x$ has a multiplicative inverse, $x^{-1}\pmod{m}$, and it is unique (mod $m$).
	\item Theorem 6.3: Let $x\geq y > 0$. Then gcd$(x, y)$ = gcd$(y, x$ mod $y)$.
	\item Theorem 7.2 [Fermat's Little Theorem]: For any prime $p$ and any $a\in\{1,2,...,p-1\}$, we have $a^{p-1}\equiv1\pmod{p}.$ (Extra: and so, $a^y = a^{y\pmod{p-1}}\pmod{p}$
	\item Extra (need proof by ???): If $ax+bm = d$, gcd$(x,m) = d, xu\equiv v\pmod{m},$ then it has a solution $\iff d\mid v$; if so, one such solution is $u\equiv\frac{va}{d}\pmod{\frac{m}{d}}$, (in essence, $a = (\frac{x}{d})^{-1}\pmod{\frac{m}{d}}$,) and there are exactly {\large$d$}-many solutions (of the form $u = \frac{va}{d} + i\cdot\frac{m}{d}\pmod{m}).$
	\item Extra (Proved in HW): If $a\equiv b\pmod{m_1\land m_2}$ and gcd$(m_1, m_2) = 1$, then $a\equiv b\pmod{m_1m_2}.$
	\item Extra (Proved in HW, FLT entended to composite): If $n = p_1p_2\cdot\cdot\cdot p_k$ where $p_i$ are distinct primes and $(p_i - 1)\mid(n - 1)\ \forall i$, then $a^{n-1}\equiv1\pmod{n}\ \forall a\in\{i\ |\ 1\leq i\leq n\ \land$ gcd$(n,i) = 1\}.$
	\item Extra: $(p-1)!\equiv(p-1)\pmod{p}$: Proof by the fact that $2,...,p-2$ will pair up with their own inverse, and only ones that map back to themselves are $1, -1$.

\end{itemize}

\end{document}
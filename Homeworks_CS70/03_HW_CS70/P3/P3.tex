\documentclass{article}
\usepackage[left=3cm, right=3cm, top=3cm]{geometry}
\usepackage{amssymb}
\begin{document}

{\Large 3 Bipartite Graph} \\[.5cm]
\textit{Proof.} To prove this, we must establish two directions: if, and only if.\\[.5cm]
\textit{Only if.} We give a proof by contradiction for the forward direction, i.e., if a graph $G$ is bipartite, then it has no tours of odd length. Assume, for a contradiction, that for a bipartite graph $G_1$, there is a tour $T$ of odd length. \\[.3cm]
\indent By definition of partite, consider the 2 disjoint sets of vertices of $G$, say $L$ and $R$, such that no 2 vertices in the same set have an edge between them. WLOG, let our tour $T$ begin in a vertex $v_0\in R$, and let our $i^{th}$ edge be $\{v_{i-1}, v_i\}$. We claim that for any odd-numbered edge, let it be the $(2n+1)^{th}$ edge $\{v_{2n}, v_{2n+1}\}$, $n\in\mathbb{N}$, then vertex $v_{2n+1}\in L.$ \\[.3cm]
\indent We proceed by induction on $n.$\\[.2cm]
\indent \textit{Base case} (n = 0): Consider edge $\{v_0, v_1\}.$ Since there must not be an edge between 2 vertices in the same set (as stated above by definition), so $v_1\in L.$\\[.2cm]
\indent Thus, the base case is correct.\\[.2cm]
\indent \textit{Inductive Hypothesis:} Assume that the claim, for the $(2n+1)^{th}$ edge $\{v_{2n}, v_{2n+1}\}$ and $n\in\mathbb{N}$, $v_{2n+1}\in L,$ is true for $n = k\in\mathbb{N}.$\\[.2cm]
\indent \textit{Inductive Step:} We prove the claim for $n = k + 1\geq1$. Since $v_{2k+1}\in L,$ and by definition of bipartite, so edge $v_{2k+1}, v_{2k+2}$ connects two vertices in different sets, which implies that $v_{2k+2}\in R.$ Thus, similarly, since $v_{2k+2}, v_{2k+3}$ is an edge, so $v_{2k+3}\in L,$ which gives us that $v_{2(k+1)+1}\in L.$\\[.2cm]
\indent Thus, by the principle of mathematical induction, the claim holds.\\[.3cm]
\indent Thus, the last vertex of the tour $T$ would always be in the set $R$, while the first vertex of the tour $T$, by assumption, is in the set $L,$ which implies that $T$ is not a tour.\\[.2cm]
\indent Therefore, there isn't a tour $T$ of odd length in a bipartite graph, so we have that if $G$ is bipartite, then $G$ has no tours of odd length. \\[.5cm]
\textit{If.} We give a Proof by Cases by providing a bipartition on the set of vertices of $G$ where $G$ is a graph with no tours of odd length, and we'll show that $G$ is bipartite. \\[.3cm]
\indent We proceed by cases. Let us divide our proof into three cases, exactly one of which must be true: (1) $G$ has no tours; or (2) $G$ has tours, but no tours of odd length. \\[.3cm]
\indent Case (1): With the assumption that $G$ has no tours, consider any path in $G,$ so it doesn't have repeating vertices (or else it would result in a sub-path that is a tour). Similarly, $G$ have no cycles. Thus, $G$ only have walks. Let $W_1, W_2, ..., W_k$ be the edge disjoint walks of $G.$ Let's start with $W_1.$ Let the sequence of vertices that $W_1$ traverse be $v_{1,1}, v_{1,2}, ..., v_{1,w1}.$ Let $L, R$ be the two sets we will be putting our vertices into, and we define them such that no 2 vertices in the same set have an edge between them. First, let $v_{1,1}\in L,$ and then alternate $R, L$ to put in the vertices $W_1$ traverses. In general, $v_{1, 2i-1}\in L, v_{1, 2j}\in R$ where $i,j\in\mathbb{Z^+}.$ Since all the edges in $W_1$ are of the form $\{v_{1, k}, v_{1, k+1}\}, k\in\mathbb{Z^+},$ so this assignment of vertices would not violate our assumption regarding $L$ and $R.$ \\[.2cm]
\indent Then, consider $W_2.$ Again, Let the sequence of vertices that $W_2$ traverse be $v_{2,1}, v_{2,2}, ..., v_{2,w2}.$ There are two situations: either no vertex in $W-2$ have already been assigned into $L$ or $R,$ or there are vertices in $W-2$ that have already been assigned into $L$ or $R.$\\[.2cm]
\indent\indent If none of the vertices in $W-2$ have already been assigned into $L$ or $R,$ then we repeat the process for vertices in $W_2$ just as we did for $W_1.$ \\[.2cm]
\indent\indent Else if some vertices in $W-2$ have already been assigned into $L$ or $R.$ We claim that there is at most 1 such vertex. We proceed by contradiction. Assume that there are two vertices $u, v$ that have already been assigned into $L$ or $R.$ Then, there is a sub-walk of $W_2$ from $u$ to $v$, and similarly, there is a sub-walk of $W_1$ from $u$ to $v,$ which contradicts our assumption that there should be no tours in $G.$ Thus, there is at most 1 vertex that have been assigned already. WLOG, assume this vertex $v_{2, i}\in L.$ So, similar to how we assigned vertices of $W_1,$ we would assign all vertices that are adjacent to $v_{2, i}$ into $R,$ then assign all vertices that are adjacent to them into $L$ again, and continue this process. This would give us a bipartition that does not violate the assumption. \\[.2cm]
\indent Now, consider $W_3.$ Let the sequence of vertices that $W_3$ traverse be $v_{3,1}, v_{3,2}, ..., v_{3,w2}.$ For situations like having zero or one vertices that have already been assigned, we will utilize a similar strategy as discussed above. Yet, there will only be one exception: suppose both $v_{3,i}\in W_i$ and $v_{3,j}\in W_j$ have already been assigned. We have proved that they could not have been assigned in the same walk $W_k,$ or else it would lead to a tour, and thus contradicting our assumption. Start with $v_{3,i}$, and WLOG, let $v_{3,i}\in L$. Label its neighbors into $R,$ and repeat this process. When we reach $v_{3_j},$ if our new assignment is the same as its previous assignment, then the algorithm would lead to a successful bipartition; on the other hand, if our new assignment is different as its previous assignment, then we would flip the assignment of the entire walk $W_j.$ This would lead to a successful bipartition as $v_{3_j}$ would now have the same assignment as its previous assignment, and the labeled $W_j$ would still be successfully bipartitioned. Continue this process for all walks $W_4, ..., W_k,$ and we will have successfully bipartitioned all the vertices of $G.$ \\[.2cm]
\indent Thus, we could always find a bipartition if $G$ has no tours, which implies that if $G$ has no tours, then $G$ is bipartite. \\[.3cm]
\indent Case (2): With the assumption that $G$ has tours and no tours of odd length, we have that all the tours $T_1, T_2, ..., T_t$ are of even length, which means that for any arbitrary tour $T_x, 1\leq x\leq t,$ and let the vertices $T_x$ traverse be $v_{x,1}, v_{x,2}, ..., v_{x,wx}, v_{x,1}$ (since a tour is closed). Since $T_x$ is of even length, so there are an even number of edges in $T_x$, denoted as $\{v_{x,1}, v_{x,2}\}, \{v_{x,2}, v_{x,3}\, ..., \{v_{x,wx-1}, v_{x,wx}\}, \{v_{x,wx}, v_{x,1}\}$. Thus, this implies that $wx$ is even, which means that we could assign the vertices of $T_x$ into disjoint sets $L$ and $R$ just like how we assigned the vertices of walks in Case (1). Thus, similar to our proof in Case (1), we have that if $G$ only has no tours of odd length, then $G$ is still bipartite.\\[.3cm]
Therefore, we have that if $G$ has no tours of odd length, then $G$ is bipartite. \\[.5cm]
Therefore, we conclude that a graph is bipartite if and only if it has no tours of odd length.\\[.5cm]
Q.E.D.


\end{document}


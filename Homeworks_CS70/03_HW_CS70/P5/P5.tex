\documentclass{article}
\usepackage[left=3cm, right=3cm, top=3cm]{geometry}
\usepackage{amssymb}
\begin{document}

{\Large 5 Triangulated Planar Graph} \\[.5cm]
(a) 12 \\[.3cm]
\indent Let the triangulated planar graph $G$ have $V$ vertices, $F$ faces, and $E$ edges. Using Euler's formula, so we have $V + F = E + 2.$\\[.1cm]
\indent Then, since every face has 3 sides, and since each edge is always shared by two faces, so $2E = 3F,$ which gives us that $E = \frac{3}{2}F.$ Thus, in this situation, we have that Eulers's formula is equivalent to $V + F = \frac{3}{2}F + 2,$ which means that $V = \frac{1}{2}F + 2.$ Moreover, consider any arbitrary vertex $v$, and let its charge be $c(v).$ By our definition of ``charge,'' $c(v)$ = 6 $ - $ degree$(v),$ which implies that $c$ = 6 $-$ the number of edges connected to $v.$ Since every edge connects two vertices, so the sum of the degrees of all vertices is twice the number of edges, or $2E.$ Thus, the sum of the charges on all the vertices, $\sum C = \sum\limits_{v\in V} c(v) = \sum\limits_{v\in V} (6$ $-$ degree$(v))$ = $\sum\limits_{v\in V} (6) -\sum\limits_{v\in V}$ degree$(v) = 6V - 2E = 6(\frac{1}{2}F + 2) - 3F = 3F + 12 - 3F = 12.$\\[.1cm]
\indent Therefore, the sum of the charges on all the vertices is 12. \\[.5cm]
(b) Charge of degree five vertex: \textbf{1}; charge of degree six vertex: \textbf{0}.\\[.3cm]
\indent Since we define charge on any vertex $v$ as the value of 6 $-$ degree$(v)$, so the charge of a degree 5 vertex is $6 - 5 = 1,$ and the charge of a degree 6 vertex is $6 - 6 = 0.$ \\[.5cm]
(c) Proof by contradiction.\\[.1cm]
\indent\textit{Proof.} We proceed by contradiction. Assume that the proposition is false, which means that there exists a triangulated planar graph $G$ such that $G$ does not have any vertex of degree 1, 2, 3, 4, and that $G$ does not have two degree 5 vertices which are adjacent, and that $G$ does not have a degree 5 vertex and a degree 6 vertex which are adjacent, under the assumption that after discharging all degree 5 vertices, there is a degree 5 vertex with positive remaining charge. Let $R$ be the assertion that there is a degree 5 vertex with positive remaining charge.\\[.1cm]
\indent Consider any degree 5 vertex $v\in G.$ By definition of the degree of a vertex, so $v$ has 5 ``neighbors.'' Since there is no vertex of 1, 2, 3 or 4, and that $v$ could not be adjacent to any degree 5 or degree 6 vertices, so for any neighbor $v_n$ of $v$, we have that $v_n$ has a degree $d\geq7.$ So, $v_n$ has a charge $c\leq 6-7 = -1.$ In other words, since $v_n$ was chosen arbitrarily, so all 5 neighbors of $v$ has a negative charge. Thus, after discharging $v,$ we have that $v$ shifted all of its charge to its neighbors, which implies that $v$ would have a charge of 0, so $v$ does not have positive remaining charge. Since we chose $v$ arbitrarily as well, so there would be no degree 5 vertex with positive remaining charge, which implies $\neg R.$\\[.1cm]
\indent We conclude that $R\land\neg R$ holds; thus, we have a contradiction, as desired. Therefore, a situation like this is impossible, which implies that the statement regarding triangulated planar graph is true under the assumption that, after discharging all degree 5 vertices, there is a degree 5 vertex with positive remaining charge.\\[.1cm]
\indent Q.E.D.\\[.5cm]
(d) \textbf{Yes}, it does. The possible degrees of that vertex are: \textbf{1, 2, 3, 4, 7}.\\[.3cm]
\indent Since we have proved earlier in part (a) that the sum of the charges on all the vertices is 12, so there must exist vertices with positive charge after discharging.\\[.1cm]
\indent For the second part of the question, WLOG, consider a vertex $v_1$ with degree 1, so $v_1$ has a charge of $6 - 1 = 5 > 0.$ Since $v_1$ was never discharged, so it remains with a charge of 5, so vertices with degree 1 would have positive charge after discharging. Similarly, since $6 - 2 = 4, 6 - 3 = 3, 6 - 4 = 2,$ so degree 2 vertices, degree 3 vertices, and degree 4 vertices would remain with a charge of 4, 3, 2, respectively after discharging, so they would all have positive charges.\\[.1cm]
\indent Degree 7 vertices could possibly have a positive charge after discharging the degree 5 vertices because consider $v_7\in G,$ a degree 7 vertex, so the initial charge of $v_7$ is $6 - 7 = -1 < 0.$ Consider this situation: let all of $v_7$'s ``neighbors,'' or all the vertices it's incident to (let them be $v_{7,1}, v_{7,2}, v_{7,3}, v_{7,4}, v_{7,5}, v_{7,6}, v_{7,7}$) are degree 5 vertices. WLOG, consider $v_{7,1},$ which has a degree of 5, so a charge of $6 - 5 = 1.$ Since $v_7$ has a negative charge and is incident to $v_{7,1}$ (in other words, it is the neighbor of $v_{7,1}$), so $v_{7,1}$ would shift 1/5 of its charge, which is $\frac{1}{5}$ to $v_7.$ Similarly, $v_{7,2}, v_{7,3}, v_{7,4}, v_{7,5}, v_{7,6}, v_{7,7}$ would also each give a charge of $\frac{1}{5}$ to $v_7,$ so $v_7$ would end up with a charge of $-1 + \frac{1}{5} * 7 = -1 + \frac{7}{5} = \frac{2}{5} > 0.$ Thus, it is possible for degree 7 vertices to have positive charges after discharging the degree 5 vertices.\\[.3cm]
\indent Now we'll show that vertices with any other degree is not possible.\\[.1cm]
\indent First of all, by definition of the triangulated planar graphs, no vertex could begin with a degree $d\leq0.$\\[.1cm]
\indent Consider degree 5 vertices, our assumption indicates that ``no degree 5 vertices have positive charge after discharging the degree 5 vertices.'' \\[.1cm]
\indent Consider any degree 6 vertices, $v_6,$ so $v_6$ begin with a charge of $6 - 6 = 0.$ By definition, a degree 5 vertice would shift its charge to a neighbor if and only if that neighbor has a negative charge. Yet, $v_6$ doesn't have a negative charge, so it wouldn't receive any charge from the discharging process. Thus, degree 6 vertices would remain with a charge of 0, so they don't have positive charges after discharging.\\[.1cm]
\indent Consider any vertex $v^*$ with a degree $d^*\geq8,$ so it has a a charge of $c^*\leq 6 - d^*.$ On the other hand, the maximum amount of charge $d^*$ could receive during the discharging of degree 5 vertices is the number of its neighbors, $d^*$, times $\frac{1}{5}$, which is the maximum charge it could receive from any of its neighbors. Thus, the maximum amount of charge $d^*$ could receive is $\frac{1}{5} d^*.$ Thus, the maximum charge $v^*$ could have in the end is $6 - d^* + \frac{1}{5}d^* = 6 - \frac{4}{5}d^* \leq 6 - \frac{4}{5}*8 = -\frac{2}{5} < 0.$ Thus, this implies that any vertex with a degree greater than or equal to 8 cannot have positive charge after discharging.\\[.1cm]
\indent Therefore, we have shown that both (1) there definitely exist vertices with positive charge after discharging and (2) the only possible degrees are 1, 2, 3, 4 and 7. \\[.5cm]
(e) It might have \textbf{6 or 7} neighbors of degree 5.\\[.3cm]
\indent Consider any arbitrary degree 7 vertex $v_7$, by our definition of charge, so it has a charge of $6 - 7 = -1.$ Let us denote the number of its neighbors of degree 5 as $n.$ Thus, as I proved in part (d) above, if $n = 7$, which means that all 7 neighbors of $v_7$ is degree 5, then $v_7$ would end up with a charge of $\frac{2}{5} > 0,$ a positive charge. Moreover, if $n = 6$, so exactly 6 of its neighbors is degree 5, then similarly, $v_7$ would receive a charge of $\frac{1}{5}$ from exactly 6 neighbors, so it would end up with a charge of $(-1) + \frac{1}{5} * 6 = \frac{1}{5} > 0,$ a positive charge. \\[.1cm]
\indent Now we'll show that if $v_7$ has any other degree, or in other words, if $n\neq6$ and $n\neq7$, then it would not have a positive charge after discharging all degree 5 vertices. \\[.1cm]
\indent First, by definition of degree, it is not possible for $v_7$ to have 8 or more neighbors of degree 5, so $n\leq8.$ Similarly, by definiton, $n\geq0,$ so we have that $0\leq n\leq 5.$ Thus, similar to our logic above, the amount of charge $v_7$ could receive from its neighbors, $c_{receive}\leq\frac{1}{5} * 5 = 1.$ Thus, the eventual charge of $v_7, c\leq (-1) + 1 = 0,$ so $v_7$ would not have a positive charge. Thus, this gives us that $v_7$ would not have a positive charge if $n\neq6$ and $n\neq7.$ \\[.5cm]
(f) Yes, they are.\\[.3cm]
\indent Again, let the graph be $G$ and consider our $v_7$ from part (e), which is a degree 7 vertex that has positive charge after discharging the degree 5 vertices. Let its neighbors be $u_1, u_2, u_3, u_4, u_5, u_6, u_7$ in clockwise order. Since we have proved in part (e) that it has either 6 or 7 neighbors of degree 5, there is at most one neighbor not of degree 5. If there is one such neighbor (not of degree 5), WLOG, let it be $u_7.$ Thus, under either situation, $u_1, u_2$ are two of $v_7$'s neighbors of degree 5 such that there is no other edges between $\{v_7, u_1\}$ and $\{v_7, u_2\}.$ \\[.1cm]
\indent Then, we claim that there must be a walk in $G$ from $u_1$ to $u_2$ that doesn't involve $v_7$ (in other words, there must be a cycle including both $u_1$ and $u_2$). We proceed by contradiction. Let $R$ be the assertion that $G$ is a triangulated planar graph. \\[.1cm]
\indent Since there is no cycle including both $u_1$ and $u_2$, so there must be a unbounded face $F^*$ bordered by $\{v_7, u_1\}$ and $\{v_7, u_2\}.$ Consider the face bordered by edges $\{v_7, u_1\}$ and $\{v_7, u_2\}$ that is not closed, so the number of edges bordering it must be at least $2 + 2 = 4 > 3,$ contradicting the definition of a triangulated planar graph, thus implying $\neg R.$ \\[.1cm]
\indent We conclude that $R\land\neg R$ holds; thus, we have a contradiction, as desired. Therefore, a situation like this is impossible, which implies that there must be a cycle including both $u_1$ and $u_2.$ \\[.1cm]
\indent Now, assume that the cycle we found including both $u_1$ and $u_2$ has greater than or equal to 4 edges, then the face bounded by this cycle would have at least 4 edges bordering it; again, this violates our definition of a triangulated planar graph, so the cycle we found must have exactly 3 edges, and since we already have that $\{v_7, u_1\}$ and $\{v_7, u_2\}$ are two edges of this cycle, so the last edge must be $\{u_1, u_2\},$ which gives us the conclusion that two of these degree 5 vertices are adjacent. \\[.1cm]
\indent Q.E.D.\\[.5cm]
(g) Proof by Cases.\\[.3cm]
\indent\textit{Proof.} We proceed by cases. Let us divide our proof into three cases, exactly one of which must be true: using the definition of ``charge'' and the process of discharging all degree 5 vertices, so (1) there is a degree 5 vertex with positive remaining charge; or (2) no degree 5 vertices remain with positive charge, and that there is a degree 7 vertex with positive charge; or (3) no degree 5 vertices remain with positive charge, and that there is no degree 7 vertex with positive charge. For clearer notation, let us denote the proposition we wish to prove as $P,$ which says that every triangulated planar graph contains either (a) a vertex of degree 1, 2, 3, 4, (b) two degree 5 vertices which are adjacent, or (c) a degree 5 and a degree 6 vertices which are adjacent. \\[.2cm]
\indent Case (1): With the assumption that there is a degree 5 vertex with positive remaining charge after discharging all degree 5 vertices, as we proved in part (c), so we have that $P$ is true. \\[.1cm]
\indent Case (2): With the assumption that no degree 5 vertices have positive charge after discharging the degree 5 vertices, and that there exists a degree 7 vertex with positive charge, as we proved in part (f), we have that two of the degree 5 neighbors of the degree 7 vertex are adjacent. Thus, with the the $2^{nd}$ condition of $P$'s conclusion satisfied, we have that $P$ is true. \\[.1cm]
\indent Case (3): With the assumption that no degree 5 vertices have positive charge after discharging the degree 5 vertices, and that there is no degree 7 vertex with positive charge. Since we proved in part (d) that if no degree 5 vertices have positive charge after discharging all the degree 5 vertices, then there must still exist vertices with positive charge, and the only possible degrees of these vertices are 1, 2, 3, 4 and 7. Since we have that there is no degree 7 vertex with positive charge after discharging, so there must be vertices with degree 1, 2, 3 or 4, thus satisfying the $1^{st}$ condition of $P$'s conclusion, so we have that $P$ is true. \\[.2cm]
\indent Thus, the proposition $P$, every triangulated planar graph contains either (a) a vertex of degree 1, 2, 3, 4, (b) two degree 5 vertices which are adjacent, or (c) a degree 5 and a degree 6 vertices which are adjacent, is true, and the proof is complete.\\[.1cm]
\indent Q.E.D.

\end{document}


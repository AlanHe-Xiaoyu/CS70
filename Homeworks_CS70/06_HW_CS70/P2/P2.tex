\documentclass{article}
\usepackage[left=3cm, right=3cm, top=3cm]{geometry}
\usepackage{amssymb}
\usepackage{amsmath}
\begin{document}

{\Large 2 The CRT and Lagrange Interpolation} \\[.5cm]
(a) Direct Proof \\[.1cm]
$\textit{Proof.}$ We proceed by a direct proof for both statements given $k = 2$. First, we show that: by assumption, $n_1, n_2$ are coprime, which is equivalent to gcd$(n_1, n_2) = 1$, so we can write $k_1n_1 + k_2n_2 = 1$ for some $k_1, k_2\in\mathbb{Z}$.
\\[.1cm]
\indent Part 1: Given that $a_1 = 1, a_2 = 0$, so we have that $x_1\equiv1\pmod{n_1}, x_1\equiv0\pmod{n_2}$.
Now, consider $x_1 = k_2n_2$. Since $n_2, k_2\in\mathbb{Z},$ so $x_1\in\mathbb{Z}$ and $x_1\equiv0\pmod{n_2}.$ Since we also have $x_1 = 1 - k_1n_1 = (-k_1)\cdot(n_1) + 1$, with $-k_1\in\mathbb{Z}$, so $x_1\equiv1\pmod{n_1}$, which implies that $x_1 = k_2n_2$ is a valid solution to the first situation. \\[.1cm]
\indent Part 2: Given that $a_1 = 0, a_2 = 1$, so we have that $x_2\equiv0\pmod{n_1}, x_2\equiv1\pmod{n_2}$.
Now, consider $x_2 = k_1n_1\in\mathbb{Z}$. Similarly, we have that $x_2\equiv0\pmod{n_1}$, and with $x_2 = 1 - k_2n_2 = (-k_2)\cdot(n_2) + 1$, so similar to Part 1 above, $x_2\equiv1\pmod{n_2}$, which gives that $x_2 = k_1n_1$ is a valid solution to the first situation. \\[.1cm]
Q.E.D. \\[.3cm]
(b) Direct Proof \\[.1cm]
$\textit{Proof.}$ We proceed by a direct proof for both statements where we still write $k_1n_1 + k_2n_2 = 1$ for some $k_1, k_2\in\mathbb{Z}$. \\[.1cm]
\indent For any given $a_1, a_2$, consider $x = a_1k_2n_2 + a_2k_1n_1$. So, $x = a_1(1 - k_1n_1) + a_2k_1n_1 = a_1 + (-a_1k_1 + a_2k_1)n_1$. Since $a_1, k_1, a_2, k_2\in\mathbb{Z},$ so $-a_1k_1 + a_2k_1\in\mathbb{Z},$ which gives us that $x\equiv a_1\pmod{n_1}$. Similarly, $x = a_1k_2n_2 + a_2k_1n_1 = a_1k_2n_2 + a_2(1 - k_2n_2) = a_2 + (a_1k_2 - a_2k_2)n_2$ and since $a_1, k_1, a_2, k_2\in\mathbb{Z},$ so $a_1k_2 - a_2k_2\in\mathbb{Z}$, so we have $x\equiv a_2\pmod{n_2}$. Thus, there exists at least one solutoin $x$ to (1) and (2) for any $a_1, a_2.$ \\[.1cm]
\indent For any two solutions $x', x^*$ to (1) and (2) with given $a_1, a_2,$ we have that $x'\equiv x^*\equiv a_1\pmod{n_1}$ and $x'\equiv x^*\equiv a_2\pmod{n_2}.$ So, $x'-x^*\equiv0\pmod{n_1}$ and $x'-x^*\equiv0\pmod{n_2}$. Since given that gcd$(n_1, n_2) = 1$, using previous homework results, we have that $x'-x^*\equiv0\pmod{n_1n_2}$. Thus, $x'\equiv x^*\pmod{n_1n_2}$, which implies that all possible solutions are equivalent $\pmod{n_1n_2}$, as desired. \\[.1cm]
Q.E.D. \\[.3cm]
(c) Direct Proof \\[.1cm]
$\textit{Proof.}$ We proceed by a direct proof for both statements. \\[.1cm]
\indent Since for all $i\neq j,$ it is given that $n_i, n_j$ are coprime, so we can repeat the process we described and proved in part (b) by solving two equations at a time, which will always yield us a solution $x$. In other words, $\exists$ a solution $x$ to (1)-($k$). \\[.1cm]
\indent Again, we can show that this solution is unique $\pmod{n_1n_2\cdot\cdot\cdot n_k}$ by showing that for any two solutions $x', x^*$ to (1)-($k$), we have $x'-x^*\equiv0\pmod{n_1n_2\cdot\cdot\cdot n_k}$ by repetitively using the strategy presented in part (b), given that for all $i\neq j,$ it is given that $n_i, n_j$ are coprime. Thus, $x'\equiv x^*\pmod{n_1n_2\cdot\cdot\cdot n_k}$, which implies that the solution $x$ is unique $\pmod{n_1n_2\cdot\cdot\cdot n_k}$. \\[.1cm]
Q.E.D. \\[.3cm]
(d) Definition see below; $p(1)$ \\[.1cm]
\indent Let $p(x) = k(x)\cdot q(x) + r(x)$ where $k(x), r(x)$ are also polynomials and $0\leq$ deg$(r) <$ deg$(q)$.
Mimicing the definition of $a$ mod $b$ for integers, we define polynomial mod, $p(x)$ mod $q(x)$, to be: $q(x)\equiv r(x)\pmod{p(x)}$ where $0\leq$ deg$(r) <$ deg$(q)$. \\[.1cm]
\indent Then, consider when $x = 1$, we have $p(x) - p(1) = p(1) - p(1) = 0$, which means that $x = 1$ is a root for $p(x) - p(1)$ where $p(1)$ is a constant that can be calculated, which implies that deg$(p(1)) = 0 < 1 = $ deg$(x-1).$ Since we also have that $p(x) - p(1) = 0\equiv0\pmod{x-1}$, so $p(x)\equiv p(1)\pmod{x-1}$. Thus, $p(x)$ mod $(x-1)$ is $p(1).$ \\[.3cm]
(e) Direct Proof; connection to Lagrange interpolation explained below. \\[.1cm]
$\textit{Proof.}$ We proceed by a direct proof for both statements. We claim that each of the $x-x_i$ are pairwise coprime given the $x_i$ are pairwise distinct. \\[.1cm]
\indent We proceed by a proof by contradiction to prove the above claim. Assume that for some two polynomials $x-x_m, x-x_n$ with $x_m\neq x_n$, they have a common divisor of degree 1, $ax+b$ and $a\neq0$. Let $R$ be the assertion that $x_m\neq x_n$ and let $x-x_m = k_m(ax+b), x-x_n = k_n(ax+b)$ where $k_m, k_n\in\mathbb{R}.$ So, $x-x_m = ak_mx + bk_m$ and $x-x_n = ak_nx + bk_n$, which gives us these four equations:
\begin{center}
$1 = ak_m$ \\
$-x_m = bk_m$ \\
$1 = ak_n$ \\
$-x_n = bk_n$
\end{center}
\indent So, we have $1 = ak_m = ak_n$. With $a\neq0$, so $k_m = k_n$. Thus, the equations above gives us that $x_m = -bk_m = -bk_n = x_n$, which implies $\neg R$. So, $R\land\neg R$ holds, which gives the contradiction. \\[.1cm]
\indent Thus, our claim is true that each of the $x-x_i$ are pairwise coprime. Then, since we're told that the CRT still holds when replacing $x, a_i, n_i$ with polynomials and using the coprime definition, so the system of congruences given has a unique solution $\pmod{(x-x_1)\cdot\cdot\cdot(x-x_k)}$ whenever the $x_i$ are pairwise distinct. \\[.3cm]
\indent Now, this is very similar to our Lagrange interpolation, since the way we write the greatest common divison of the $(x-x_i)$'s corresponds to the first step of finding the polynomials $
\Delta_i$, and finding the actual solution $p(x)$ by multiplication corresponds to our step in CRT where we multiply the base solution by $y_i$ for each corresponding factor. Therefore, using the CRT for polynomials to find $p(x)$ is an equivalent method to Lagrange interpolation. Thus, this is also another proof of why Lagrange interpolation works and why there's a unique solution $p(x)$ in the modular setting, correspondingly, $GF(p).$


\end{document}
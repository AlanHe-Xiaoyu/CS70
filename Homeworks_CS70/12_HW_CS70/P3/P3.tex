\documentclass{article}
\usepackage[left=3cm, right=3cm, top=3cm]{geometry}
\usepackage{amssymb}
\usepackage{amsmath}
\usepackage{mathrsfs}
\usepackage{xcolor}
\begin{document}

{\Large 3 Geometric and Poisson} \\[.5cm]
{\color{red} (a) $e^{-\lambda p}$} \\

Since $X\sim\text{Geo}(p), Y\sim\text{Poisson}(\lambda)$, so we have that
$\mathbb{P}(X>Y) = 
\sum\limits_{i=0}^\infty \mathbb{P}[Y=i]\mathbb{P}[X>i]$.
Now, for any $i\in\mathbb{N}$, we have that
$$\mathbb{P}[Y=i] = \frac{\lambda^i}{i!}e^{-\lambda}$$

and with $p\neq 0$, we have 
$$\mathbb{P}[X>i] = \sum\limits_{j=i+1}^\infty \mathbb{P}[X=j] =
\sum\limits_{j=i+1}^\infty (1-p)^{j-1}p =
\frac{(1-p)^i p}{1-(1-p)} = (1-p)^i$$

Then, using the Taylor series expansion of $e^x$, we have that:
$$\mathbb{P}(X>Y) = \sum\limits_{i=0}^\infty \frac{\lambda^i}{i!}e^{-\lambda} \cdot (1-p)^i =
e^{-\lambda} \cdot \sum\limits_{i=o}^\infty \frac{(\lambda (1-p))^i}{i!} =
e^{-\lambda} \cdot e^{\lambda(1-p)}$$

Thus,
$$\mathbb{P}(X>Y) = e^{-\lambda + \lambda - \lambda p} = e^{-\lambda p}$$ \\[.5cm]
{\color{red} (b) 1} \\

Since $Z$ is defined as $Z = \text{max}(X, Y)$, so by definition, we have that $\forall i,\ Z\geq X$. Therefore,
$\mathbb{P}(Z\geq X) = 1$. \\[1cm]
{\color{red} (c) $1 - e^{-\lambda p}$} \\

Again, since $Z = \text{max}(X, Y)$, so we have that
$\mathbb{P}(Z\leq Y) = \mathbb{P}(Z=Y) = \mathbb{P}(X\leq Y) =
\mathbb{P}(\overline{X>Y}) = 1 - \mathbb{P}(X>Y)$. Using our result from part (a), so:
$$\mathbb{P}(Z\leq Y) = 1 - \mathbb{P}(X>Y) = 1 - e^{-\lambda p}$$


\end{document}
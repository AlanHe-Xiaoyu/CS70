\documentclass{article}
\usepackage[left=3cm, right=3cm, top=3cm]{geometry}
\usepackage{amssymb}
\begin{document}

{\Large 5 FLT Converse} \\[.5cm]
(a) Direct proof \\[.3cm]
\textit{Proof.} We proceed by a direct proof. For any $a$ that's not relatively prime to $n,$ let gcd$(a, n) = d, d > 1.$ Let $a = d\cdot a^*, n = d\cdot n^*.$ For any $x\in\mathbb{Z},$ let $ax = kn + y, k\in\mathbb{Z}, 0\leq y<n.$ By definition of modular arithmetic, we have that $ax\equiv y\pmod{n}.$ Substituting, we have that $d\cdot a^*\cdot x = k\cdot d\cdot n^* + y.$ Since $d\mid (d\cdot a^*\cdot x),$ so $d\mid (k\cdot d\cdot n^* + y),$ so we can infer that $d\mid y.$ Since $d > 1,$ so $y > 1,$ and since $0\leq y<n,$ so we have proved that any multiple of $a$ mod $n$ would not be 1. Thus, $a^{n-1} = a\cdot a^{n-2}\not\equiv1\pmod{n}.$ \\[.1cm]
Q.E.D.\\[.5cm]
(b) Direct proof \\[.3cm]
\textit{Proof.} We proceed by a direct proof. Suppose we have some $a\in S(n)$ such that $a^{n-1}\not\equiv1\pmod{n}.$ Consider any $x\in S(n)$ such that $x^{n-1}\equiv1\pmod{n},$ we claim that for $k\equiv ax\pmod{n}$ where $1\leq k\leq n,$ we have that $k$ is another such $a,$ i.e. $k\in S(n)$ and $k^{n-1}\not\equiv1\pmod{n}.$ \\[.1cm]
\indent First, we'll show that $k\in S(n).$ On the one hand, we have defined that $1\leq k\leq n.$ On the other hand, since $a,x\in S(n),$ so we have that gcd$(n, a) = 1$ and gcd$(n, x) = 1,$ which would give us that gcd$(n, ax) = 1,$ so gcd$(n, k) = 1$. Thus, by definition of the set $S(n),$ so $k\in S(n).$ \\[.1cm]
\indent Then, since $a^{n-1}\not\equiv1\pmod{n},\ x^{n-1}\equiv1\pmod{n}$ and $k\equiv ax\pmod{n}$, so we have that  $k^{n-1}\equiv(ax)^{n-1} = a^{n-1}\cdot x^{n-1}\equiv a^{n-1}\cdot1 = a^{n-1}\not\equiv1\pmod{n}.$ \\[.1cm]
Thus, we have proved if we can find some $a\in S(n)$ such that $a^{n-1}\not\equiv1\pmod{n}$, then for any $x\in S(n)$ such that $x^{n-1}\equiv1\pmod{n},$ we have a corresponding $k\in S(n)$ such that $k^{n-1}\not\equiv1\pmod{n}.$ Also note that since gcd$(a, n) = 1,$ so for any two different $x\in S(n)$ (namely, $1\leq x\leq n$), then $ax$ is unique mod $n,$ which implies that the $k$ we constructed would be unique for different $n$. In other words, we have an injection from the set of numbers in $S(n)$ that pass the FLT condition to the set of numbers in $S(n)$ that fail it. This implies that the set of numbers in $S(n)$ that fail the FLT condition is at least as large as the set of numbers in $S(n)$ that pass it. \\[.1cm]
Therefore, we have that if we can find a single $a\in S(n)$ such that $a^{n-1}\not\equiv1\pmod{n}$, then we can find at least $|S(n)| / 2$ such $a.$ \\[.1cm]
Q.E.D. \\[.5cm]
(c) Direct proof \\[.3cm]
\textit{Proof.} We proceed by a direct proof. Let $a, b, m_1, m_2\in\mathbb{Z}$ such that $a\equiv b\pmod{m_1}, a\equiv b\pmod{m_2}$ and gcd$(m_1, m_2) = 1.$ So, we have that $(a-b)\equiv0\pmod{m_1}$ and $(a-b)\equiv0\pmod{m_2},$ which is equivalent to $m_1\mid(a-b)$ and $m_2\mid(a-b).$ \\[.1cm]
So, let $(a - b) = m_1k$ where $k\in\mathbb{Z},$ and we have that $m_2\mid(m_1k).$ And since we are given that gcd$(m_1, m_2) = 1,$ so we have $m_2\mid k.$ Let $k = m_2k^*, k^*\in\mathbb{Z}.$ So, $(a-b) = m_1k = m_1m_2k^*$ is a multiple of $m_1m_2.$ In other words, $(m_1m_2\mid(a-b),$ which implies that $a\equiv b\pmod{m_1m_2},$ as desired. \\[.1cm]
Q.E.D. \\[.5cm]
(d) Direct proof \\[.3cm]
\textit{Proof.} We proceed by a direct proof. Let $n = p_1p_2\cdot\cdot\cdot p_k$ where $p_i$ are distinct primes and $(p_i-1)\mid(n-1)$ for all $i, 1\leq i\leq k.$ Let $a$ be an arbitrary element such that $a\in S(n).$ Thus, by our definition of $S(n),$ we have that gcd$(n, a) = 1.$ We claim that $a$ is not the multiple of any $p_x$ where $1\leq x\leq k.$ \\[.1cm]
\indent Suppose, for a contradiciton, that for some $1\leq x\leq k,$ we have $p_x\mid a.$ Since we also have that $p_x\mid n,$ so $p_x$ is a common divisor for $a, n$, which means that gcd$(n, a) > p_i > 1,$ which causes a contradiction. \\[.1cm]
Thus, we have proved that $a$ is not the multiple of any $p_x$ where $1\leq x\leq k.$ Thus, for any $1\leq j\leq k,$ so $a$ is coprime with any $p_j.$ Let $a\equiv a_j\pmod{p_j}$, so $1\leq a_j\leq(p_j - 1).$ Using Fermat's Little Theorem, so we have that $a^{p_j - 1}\equiv a_j^{p_j - 1}\equiv1\pmod{p_j}.$ \\[.1cm]
Since $1\leq j\leq k,$ so we know that $(p_j-1)\mid(n-1).$ Let $n - 1 = d_j(p_j - 1),\ d_j\in\mathbb{Z}$. Thus, $a^{n-1} = a^{d_j(p_j - 1)} = (a^{p_j - 1})^{d_j}\equiv1^{d_j}\equiv1\pmod{p_j}.$ Since $p_j$ is picked arbitrarily, so we could reduce that for all $1\leq j\leq k,$ we have that $a^{n-1}\equiv1\pmod{p_j}.$ Then, since we have that $p_i$ are distinct primes, for any two $p_p, p_q$ where $1\leq p,q\leq k,$ so gcd$(p_p, p_q) = 1.$ Then, since we just proved that $a^{n-1}\equiv1\pmod{p_p}$ and $a^{n-1}\equiv1\pmod{p_q}$, using the result of part (c) above, so we have that $a^{n-1}\equiv1\pmod{p_pp_q}.$ Thus, repeat this process for all $p_i$ where $1\leq i\leq k,$ so we have that $a^{n-1}\equiv1\pmod{p_1p_2\cdot\cdot\cdot p_k}.$ Since $n = p_1p_2\cdot\cdot\cdot p_k$, so this is equivalent to $a^{n-1}\equiv1\pmod{n}$ for all $a\in S(n)$. \\[.1cm]
Q.E.D. \\[.5cm]
(e) Direct proof \\[.3cm]
\textit{Proof.} We proceed by a direct proof. Using prime factorization, so $561 = 3\cdot11\cdot17.$ Since we also have that $3 - 1 = 2, 11 - 1 = 10, 17 - 1 = 16, 561 - 1 = 560,$ and that $560 = 280\cdot2 = 56\cdot10 = 35\cdot16,$ so we have that $(3-1)\mid(561-1), (11-1)\mid(561-1)$, and $(17-1)\mid(561-1)$. \\[.1cm]
Thus, for all $x\in S(561)$, which is equivalent to $x$ being coprime with 561 and $1\leq x\leq561,$ using our result from part (d), we have that $x^{560}\equiv1\pmod{561}.$ \\[.1cm]
Thus, for all $a$ that is coprime with 561, let $a\equiv a_{mod}\pmod{561}$ where $1\leq a_{mod}\leq561.$ Then, using our proof for the Euclidean algorithm, we have that gcd$(561, a_{mod}) = $ gcd$(561, a) = 1.$ Thus, by definition, we have that $a_{mod}\in S(561).$ So, $a^{560}\equiv a_{mod}^{560}\equiv1\pmod{561}.$ \\[.1cm]
Therefore, we have shown that for all $a$ coprime with 561, $a\equiv1\pmod{561}.$ \\[.1cm]
Q.E.D.

\end{document}
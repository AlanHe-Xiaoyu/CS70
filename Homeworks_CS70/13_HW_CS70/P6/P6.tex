\documentclass{article}
\usepackage[left=3cm, right=3cm, top=3cm]{geometry}
\usepackage{amssymb}
\usepackage{amsmath}
\usepackage{mathrsfs}
\usepackage{xcolor}
\begin{document}

{\Large 6 Analyze a Markov Chain} \\[.5cm]
{\color{red} (a) Direct Proof } \\

Given this case with $a, b \in (0,1)$, so the Markov chain can go from state 0 to state 0 in any of the $n\geq1$ steps, i.e. $n = \{1,2,3,\dots\}$, so we have
$$d(0) = gcd\{1,2,3,\cdots\} = 1$$

Similarly, we have that $d(1) = 1$ and $d(2) = 2$. Therefore,
$$d(i) = 1\ \forall\ i\in\mathscr{X}$$

which implies, by definition, that the given Markov chain is aperiodic. \\

Q.E.D. \\



\noindent{\color{red} (b)
$a(1-b)(1-a)a = a^2(1-a)(1-b)$ } \\

Given that $X(0) = 0$, so
$$\mathbb{P}[X(1)=1\ |\ X(0)=0] = a$$

and similarly
$$\mathbb{P}[X(2)=0\ |\ X(1)=1] = 1-b $$
$$\mathbb{P}[X(3)=0 | X(2)=0] = 1-a $$
$$\mathbb{P}[X(4)=1\ |\ X(3)=0] = a $$

Thus, we have that
$\mathbb{P}[X(1)=1,X(2)=0,X(3)=0,X(4)=1\ |\ X(0) = 0] = \\
\mathbb{P}[X(1)=1\ |\ X(0)=0] \cdot
	\mathbb{P}[X(2)=0\ |\ X(1)=1] \cdot
	\mathbb{P}[X(3)=0 | X(2)=0] \cdot
	\mathbb{P}[X(4)=1\ |\ X(3)=0]$
$$\Longrightarrow
\mathbb{P}[X(1)=1,X(2)=0,X(3)=0,X(4)=1\ |\ X(0) = 0] =
a(1-b)(1-a)a = a^2(1-a)(1-b)$$



\noindent{\color{red} (c) $\pi =
(\frac{1 - b}{ab + a - b + 1},
\frac{a}{ab + a - b + 1},
\frac{ab}{ab + a - b + 1})$ } \\

First, we calculate that the transition probability matrix for the given Markov chain is:
$$ \mathbf{P} =
\begin{bmatrix}
	P_{00} & P_{01} & P_{02} \\
	P_{10} & P_{11} & P_{12} \\
	P_{20} & P_{21} & P_{22}
\end{bmatrix} =
\begin{bmatrix}
	1-a & a & 0 \\
	1-b & 0 & b \\
	0 & 1 & 0
\end{bmatrix} $$

Then, we have that the balance equations can be set up with $\pi\mathbf{P} = \pi$, and so
$$[\pi_0, \pi_1, \pi_2] = [\pi_0, \pi_1, \pi_2] \cdot
\begin{bmatrix}
	1-a & a & 0 \\
	1-b & 0 & b \\
	0 & 1 & 0
\end{bmatrix} $$

Then, since the components of $\pi$ sum up to one, so we have 4 linear equations in total:
$$(1-a)\pi_0 + (1-b)\pi_1 + 0\cdot\pi_2 = \pi_0 $$
$$a\cdot\pi_0 + 0\cdot\pi_1 + 1\cdot\pi_2 = \pi_1 $$
$$0\cdot\pi_0 + b\cdot\pi_1 + 0\cdot\pi_2 = \pi_2 $$
$$\pi_0 + \pi_1 + \pi_2 = 1 $$

We could then solve them to get the invariant distribution as:
$$\pi_0 = \frac{1 - b}{ab + a - b + 1}$$
$$\pi_1 = \frac{a}{ab + a - b + 1}$$
$$\pi_2 = \frac{ab}{ab + a - b + 1}$$



\noindent{\color{red} (d) $\frac{ab - b + 1}{ab}$ } \\

Since $X(0) = 1$, so $\mathbb{P}[X(1)=2 | X(0) = 1] = b,
\mathbb{P}[X(1)=1 | X(0) = 1] = 0$, and
$\mathbb{P}[X(1)=0 | X(0) = 1] = 1-b$.
Now, since we want the expectation of the number of steps until we transit to state 2 for the first time, so we need to further examine the condition of $X(1) = 0$. \\

Since for $i\in\mathbb{N}$, we have $\mathbb{P}[X(i+1) = 0 | X(i) = 0] = 1-a$,
$\mathbb{P}[X(i+1) = 1 | X(i) = 0] = a$ and
$\mathbb{P}[X(i+1) = 2 | X(i) = 0] = 0$,
so we have that \\
\indent $\mathbb{E}[T_2|X(0)=1] = \\
\indent \mathbb{P}[X(1) = 0 | X(0) = 0] \cdot \mathbb{E}[X(1) = 0 | X(0) = 0] +
	\mathbb{P}[X(1) = 1 | X(0) = 0] \cdot \mathbb{E}[X(1) = 1 | X(0) = 0] +
	\mathbb{P}[X(1) = 2 | X(0) = 0] \cdot \mathbb{E}[X(1) = 2 | X(0) = 0] = \\
\indent (1-a) \cdot (1 + \mathbb{E}[T_2\ |\ X(0)=1]) +
	a \cdot (1 + \mathbb{E}[T_2\ |\ X(0)=1]) $ \\

which can be simplified a bit to:
$$\mathbb{E}[T_2\ |\ X(0)=1] =
\mathbb{E}[T_2\ |\ X(0)=1] + \frac{1}{a} $$

Thus, we can combine this result with our previous setup to get:
$$\mathbb{E}[T_2\ |\ X(0)=1]) =
b \cdot 1 + 0 +
	(1-b) \cdot \big(\mathbb{E}[T_2\ |\ X(0)=1] + \frac{1}{a}\big)$$
$$\Longrightarrow \mathbb{E}[T_2\ |\ X(0)=1]) =
\frac{ab - b + 1}{ab} $$

\end{document}
\documentclass{article}
\usepackage[left=3cm, right=3cm, top=3cm]{geometry}
\usepackage{amssymb}
\usepackage{amsmath}
\usepackage{mathrsfs}
\usepackage{xcolor}
\begin{document}

{\Large 5 Markov Chain Terminology} \\[.5cm]
{\color{red} (a)
Irreducible: $a\neq 0 \land b\neq 0$;
Reducible: $a = 0 \lor b = 0$ } \\

For the given Markov chain to be irreducible, it has to be able to transform between the two states, which means that the constraint is:
$$a\neq 0\ \land\ b\neq 0$$

Since the term reducible Markov chains is the complement of irreducible Markov chains, so we have that for the Markov chain to be reducible, we have:
$$a = 0\ \lor\ b = 0$$



\noindent{\color{red} (b) Direct Proof } \\

Given this case with $a = b = 1$, so the Markov chain can go from state 0 to state 0 in $n$ steps for all $n\in \{2,4,6,8,\dots\}$, so we have
$$d(0) = gcd\{2,4,\cdots\} = 2 \neq 1$$

Thus, by definition of periodicity, we have that the given Markov chain is periodic. \\

Q.E.D. \\



\noindent{\color{red} (c) Direct Proof } \\

Given this case with $a, b \in (0,1)$, so the Markov chain can go from state 0 to state 0 in any of the $n\geq1$ steps, i.e. $n = \{1,2,3,\dots\}$, so we have
$$d(0) = gcd\{1,2,3,\cdots\} = 1$$

Similarly, we have that $d(1) = 1$. Therefore,
$$d(i) = 1\ \forall\ i\in\mathscr{X}$$

which implies, by definition, that the given Markov chain is aperiodic. \\

Q.E.D. \\



\noindent{\color{red} (d)
$ \mathbf{P} =
\begin{bmatrix}
	1-b & b \\
	a & 1-a
\end{bmatrix} $ } \\

We have that the transition probability matrix for the given Markov chain is:
$$ \mathbf{P} =
\begin{bmatrix}
	P_{00} & P_{01} \\
	P_{10} & P_{11}
\end{bmatrix} =
\begin{bmatrix}
	1-b & b \\
	a & 1-a
\end{bmatrix} $$ 



\noindent{\color{red} (e)
$\pi_0 = \frac{a}{a+b},\ \pi_1 = \frac{b}{a+b}$ } \\

We have that the balance equations can be set up with $\pi\mathbf{P} = \pi$, and so
$$[\pi_0, \pi_1] = [\pi_0, \pi_1] \cdot
\begin{bmatrix}
	1-b & b \\
	a & 1-a
\end{bmatrix}$$

which gives us two linear equations, and then since there's also the condition that the components of $\pi$ sum up to one, so we have 3 linear equations in total:
$$\pi_0 = \pi_0 (1-b) + \pi_1 a$$
$$\pi_1 = \pi_0 b + \pi_1 (1-a)$$
$$\pi_0 + \pi_1 = 1$$

We could then solve them as:
$$\pi_0 = \frac{a}{a+b}$$
$$\pi_1 = \frac{b}{a+b}$$

\end{document}
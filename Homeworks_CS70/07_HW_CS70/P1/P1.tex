\documentclass{article}
\usepackage[left=3cm, right=3cm, top=3cm]{geometry}
\usepackage{amssymb}
\usepackage{amsmath}
\usepackage{mathrsfs}
\usepackage{xcolor}
\begin{document}

I worked alone without getting any help, except asking questions on Piazza and reading the Notes of this course. \\[1cm]
{\Large 1 Bijective or not?} \\[.5cm]
(a) \\[.1cm]
(i) {\color{red} Yes} \\[.2cm]
\indent Proof (one-to-one): Suppose $f(x) = f(y)$, then $10^{-5}x = 10^{-5}y$. Since $10^{-5}\neq0$, so we can divide both sides by $10^{-5}\neq0$, which gives us that $x = y$. So, $f(x) = f(y)\implies x = y$, so $f:\mathbb{R}\rightarrow\mathbb{R}$ is injective. \\[.1cm]
\indent Proof (onto): If $y\in\mathbb{R}$, then $f(10^5y) = 10^{-5}10^5y = y$. With $y\in\mathbb{R},$ so $10^5y\in\mathbb{R}$, which means that $y$ has a pre-image. Thus every $y\in\mathbb{R}$ has a pre-image, so $f:\mathbb{R}\rightarrow\mathbb{R}$ is onto. \\[.1cm]
\indent Thus, $f:\mathbb{R}\rightarrow\mathbb{R}$ is both one-to-one and onto, so it is a bijection. \\[.3cm]
(ii) {\color{red} No} \\[.2cm]
\indent We proceed by providing a counterexample to show that $f:\mathbb{Z}\cup\{\pi\}\rightarrow\mathbb{R}$ is not onto, which implies that it is not a bijection. \\[.1cm]
\indent Consider $y = 0.123456$, where $y\in\mathbb{R}$, so $y$ is in the range. Suppose, for a contradiction, that some $x_y\in\mathbb{Z}\cup\{\pi\}$ such that $f(x_y) = y$. Let $A$ be the assertion that $x_y\in\mathbb{Z}\cup\{\pi\}$.
So, $10^{-5}x_y = y = 0.123456$. So, multiply both sides $10^5$, $10^510^{-5}x_y = 10^5\cdot0.123456$, which gives us that $x_y = 12345.6$. However, we know that $x_y\not\in\mathbb{Z}$ and that $x_y\not\in\pi$, so we have $x_y\not\in\mathbb{Z}\cup\{\pi\}$, which implies that $\neg A$. So, $A\land\neg A$ holds, which gives the contradiction. \\[.1cm]
\indent Thus, $f:\mathbb{Z}\cup\{\pi\}\rightarrow\mathbb{R}$ is not onto, which implies that it is not a bijection. \\[.5cm]
(b) \\[.1cm]
(i) {\color{red} No} \\[.2cm]
\indent Consider $p = 7$, and then consider $x_1 = 2, x_2 = 3$, which are both in the domain of $f:\mathbb{N}\setminus\{0\}\rightarrow\{0,...,p\}$ with $2,3\in\mathbb{N}\setminus\{0\}$. However, with $x_1 = 2, x_2 = 3$, so $f(x_1) = p = 7 = 2\cdot3+1\equiv1\pmod{2}$ and $f(x_2) = p = 7 = 2\cdot3+1\equiv1\pmod{3}$, which means that $f(x_1) = f(x_2)$ while $x_1\neq x_2.$ \\[.1cm]
\indent Thus, $f:\mathbb{N}\setminus\{0\}\rightarrow\{0,...,p\}$ is not one-to-one by definition, so it's not a bijection. \\[.3cm]
(ii) {\color{red} Yes} \\[.2cm]
\indent First, since $p>2$ is prime, so $p$ is an odd number, and so $(p+1)/2,(p-1)/2\in\mathbb{Z}$.
Then, for any arbitrary $x$ in the domain of $f:\{(p+1)/2,...,p\}\rightarrow\{0,...,(p-1)/2\}$, we have $(p+1)/2\leq x\leq p$.
So, $0\leq(p-x)\leq\frac{p-1}{2} < x$, and so $p-x\in\{0,...,(p-1)/2\}$, which means that $p-x$ is the only solution.
Thus, $f(x) = p$ mod $x = p-x$. \\[.1cm]
\indent Proof (one-to-one): Suppose $f(x_1) = f(x_1)$, then using our deduction above, so $p - x_1 = p - x_2$.
Add $(-p + x_1 + x_2)$ to both sides and we have that $x_2 = x_1$.
So $f(x_1) = f(x_2)\implies x_1 = x_2$, so $f:\{(p+1)/2,...,p\}\rightarrow\{0,...,(p-1)/2\}$ is injective. \\[.1cm]
\indent Proof (onto): Now, if $y\in\{0,...,(p-1)/2\}$, then by our deductions above again, so $f(p-y) = p - (p-y) = y$.
With $0\leq y\leq\frac{p-1}{2}$, so $\frac{p+1}{2}\leq(p-y)\leq p$, which means that $(p-y)\in\{(p+1)/2,...,p\}$; in other words, $(p-y)$ is in the domain of $f:\{(p+1)/2,...,p\}\rightarrow\{0,...,(p-1)/2\}$, which means that $y$ has a pre-image.
Thus, every $y\in\{0,...,(p-1)/2\}$ has a pre-image, so $f:\{(p+1)/2,...,p\}\rightarrow\{0,...,(p-1)/2\}$ is onto. \\[.1cm]
\indent Thus, $f:\{(p+1)/2,...,p\}\rightarrow\{0,...,(p-1)/2\}$ is both one-to-one and onto, so it is a bijection. \pagebreak\\
(c) {\color{red} No} \\[.3cm]
\indent Since the domain $D$ is defined as $D = \{0,...,n\}$, so its cardinality is $|D| = n+1$, which is finite. Then, since the range is $\mathscr{P}(D)$, using Note 10, so its cardinality is $|\mathscr{P}(D)| = 2^{|D|} = 2^{n+1} > n+1 = |D|$ for all $n\in\mathbb{N}$. We will do a short proof by induction for the claim that $2^{n+1} > n+1\ \forall n\in\mathbb{N}.$ \\

Base case $(n = 0)$: $2^1 = 2 > 1$, so the base case is correct. \\
\indent Induction Hypothesis: For $n = k\geq0$, $2^{k+1} > k+1$ \\
\indent Inductive Step: Consider $n = k+1\geq1$, so $2^{n+1} = 2^{k+2} = 2\cdot2^{k+1}$. Then, using our induction hypothesis, so $2^{k+2} = 2\cdot2^{k+1} > 2\cdot(k+1) = 2k + 2\geq k+2 = (k+1)+1$, as desired. \\
\indent Thus, by the principal of mathematical induction, we have $2^{n+1} > n+1\ \forall n\in\mathbb{N}.$ \\

Thus, the cardinality of the domain of $f:D\rightarrow\mathscr{P}(D)$ is strictly smaller than its range, which means that $f:D\rightarrow\mathscr{P}(D)$ cannot be surjective. We'll insert a small proof by contradiction to for this claim. \\

Let $R$ be the assertion that $|\mathscr{P}(D)| > |D|$. Assume that $f:D\rightarrow\mathscr{P}(D)$ is surjective, then there must exist a function $g:\mathscr{P}(D)\rightarrow D$ that's injective, which indicates that $|\mathscr{P}(D)| \leq |D|$, which implies $\neg R$, so $R\land\neg R$ holds, which raises the contradiction. \\

Therefore, $f:D\rightarrow\mathscr{P}(D)$ is not surjective,
and thus, it cannot be bijective. \\[.5cm]
(d) {\color{red} Yes} \\[.3cm]
\indent Since $X = 1234567890$, so $X$ does not have any repeating digits. Then, since $X'$ is obtained by randomly shuffling $X$, so $X'$ have the same set of digits as $X$, and $X'$ does not have any repeating digits. \\

Proof (one-to-one): Suppose $f(x) = f(y)$ with $x, y\in\{0,...,9\}$, then the $(x+1)^{th}$ digit of $X'$ is the same as the $(y+1)^{th}$ digit of $X'$. Since we have shown earlier that $X'$ does not have any repeating digits, so this means that $x+1 = y+1$, adn so we have $x = y$. So, $f(x) = f(y)\implies x = y$, so $f:\{0,...,9\}\rightarrow\{0,...,9\}$ is injective. \\

Proof (onto): If $y\in\{0,...,9\}$ is in the range of $f$, then $y$ is a digit of the original $X$, and thus, $y$ is a digit of $X'$ by assumption of $X'$. Suppose $y$ is the $k^{th}$ digit of $X'$, so we have that $k\in\mathbb{Z}, 1\leq k\leq 10$. Consider $f(k-1)$, with $0\leq(k-1)\leq9$, which means that $k-1\in\{0,...,9\}$ is in the domain. Then, we also have that $f(k-1) = $ the $(k-1+1)^{th} = k^{th}$ digit of $X'$, which by our assumption, is equal to $y$. So, $f(k-1) = y$, which means that $y$ has a pre-image. Thus every $y\in\{0,...,9\}$ has a pre-image, so $f:\{0,...,9\}\rightarrow\{0,...,9\}$ is onto. \\

\indent Thus, $f:\{0,...,9\}\rightarrow\{0,...,9\}$ is both one-to-one and onto, so it is a bijection. \\[.3cm]

\end{document}
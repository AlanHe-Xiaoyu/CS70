\documentclass{article}
\usepackage[left=3cm, right=3cm, top=3cm]{geometry}
\usepackage{amssymb}
\usepackage{amsmath}
\usepackage{xcolor}
\begin{document}

{\Large 3 Old secrets, new secrets} \\[.5cm]
Bob$_1$ can achieve this by giving out {\color{red} $p'(1) = \frac{s'-s}{n+1} + p(1)$} instead of his actual number $p(1)$ when the Bobs gather to jointly discover the secret. \\

Consider the Lagrange interpolation process the Bobs would use once they gather together: \\

\indent\indent $\Delta_1(x) = \frac{(x-2)(x-3)\cdot\cdot\cdot(x-(n+1))}{(1-2)(1-3)\cdot\cdot\cdot (1-(n+1))} \\[.1cm]
\indent\indent \Delta_2(x) = \frac{(x-1)(x-3)(x-4)\cdot\cdot\cdot(x-(n+1))}{(2-1)(2-3)(2-4)\cdot\cdot\cdot (2-(n+1))} \\[.3cm]
\indent\indent\cdot\cdot\cdot\indent\cdot\cdot\cdot \\[.3cm]
\indent\indent \Delta_n(x) = \frac{(x-1)(x-2)\cdot\cdot\cdot(x-(n-1))(x-(n+1))}
{(n-1)(n-2)\cdot\cdot\cdot (n-(n-1))(n-(n+1))} \\[.1cm]
\indent\indent \Delta_{n+1}(x) = \frac{(x-1)(x-2)\cdot\cdot\cdot(x-n)}
{(n+1-1)(n+1-2)\cdot\cdot\cdot (n+1-n)} \\[.3cm] $
where the $\Delta_i(x) = \frac{(x-1)(x-2)\cdot\cdot\cdot
(x-(i-1))(x-(i+1))\cdot\cdot\cdot (x-n)(x-(n+1))}
{(i-1)(i-2)\cdot\cdot\cdot (i-(i-1))(i-(i+1))\cdot\cdot\cdot (i-n)(i-(n+1))} $ \\

Thus, the original polynomial $f(x)$ is: \\[.3cm]
\indent\indent $p(x) = p(1)\cdot\Delta_1(x) + p(2)\cdot\Delta_2(x) + p(3)\cdot\Delta_3(x) + \cdot\cdot\cdot + p(n+1)\cdot\Delta_{n+1}(x)$ \\[.3cm]
which gives that the secret $s = p(0) = p(1)\cdot\Delta_1(0) + p(2)\cdot\Delta_2(0) + \cdot\cdot\cdot + p(n+1)\cdot\Delta_{n+1}(0)$ \\

Now, suppose Bob$_1$ wants to trick the other Bobs into believing that the secret is actually some fixed $s'.$ Since the only thing he could lie about is $p(1)$, so let him say that he got the number $p'(1).$
Using Lagrange interpolation again, the $\Delta$'s would remain the same, and so new altered polynomial would be calculated as: \\[.3cm]
\indent\indent $p'(x) = p'(1)\cdot\Delta_1(x) + p(2)\cdot\Delta_2(x) + p(3)\cdot\Delta_3(x) + \cdot\cdot\cdot + p(n+1)\cdot\Delta_{n+1}(x)$ \\[.3cm]
which means that the new secret $s' = p'(0) = p'(1)\cdot\Delta_1(0) + p(2)\cdot\Delta_2(0) + \cdot\cdot\cdot + p(n+1)\cdot\Delta_{n+1}(0)$ \\

So, we have that: \\[.3cm]
\indent\indent $s' - s =
\Big(p'(1)\cdot\Delta_1(0) + p(2)\cdot\Delta_2(0) + \cdot\cdot\cdot + p(n+1)\cdot\Delta_{n+1}(0)\Big) - 
\Big(p(1)\cdot\Delta_1(0) + p(2)\cdot\Delta_2(0) + \cdot\cdot\cdot + p(n+1)\cdot\Delta_{n+1}(0)\Big) = 
p'(1)\cdot\Delta_1(0) - p(1)\cdot\Delta_1(0) = \big(p'(1) - p(1)\big)\cdot\Delta_1(0)$ \\[.3cm]
because we can cancel out all the other terms. Revisiting our definitions of $p'(1)$ and $p(1)$, so we have: \\[.3cm]
\indent\indent $s' - s = \big(p'(1) - p(1)\big)\cdot\Delta_1(0)$ \\

Then, since $\Delta_1(0) = \frac{(0-2)(0-3)\cdot\cdot\cdot(0-(n+1))}{(1-2)(1-3)\cdot\cdot\cdot (1-(n+1))} = 
\frac{-2\cdot-3\cdot\cdot\cdot-n\cdot-(n+1)}{-1\cdot-2\cdot-3\cdot\cdot\cdot-(n-1)\cdot-n}$, which can be canceled out (as all terms are non-zero) into the form $\Delta_1(0) = \frac{-(n+1)}{-1} = n+1$. Thus, since $n+1\neq0$ by assumption, so we can divide both sides of the equation by $\Delta_1(0) = n + 1$, which gives us that: $p'(1) - p(1) = \frac{s'-s}{n+1}$, which would then allow Bob$_1$ to calculate: \\[.3cm]
\indent\indent {\color{red} $p'(1) = \frac{s'-s}{n+1} + p(1)$} \\

Thus, with $s$ and $(n+1)$ being known by Bob$_1$ and $s'$ being his goal in mind, so he can decide his fake value $p'(1)$ with the equation above and trick the other Bobs into believing that secret is actually some fixed $s'$ instead of the original $s$.

\end{document}
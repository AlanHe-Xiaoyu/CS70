\documentclass{article}
\begin{document}

I worked alone without getting any help. \\[.3cm]

{\Large 1 Administrivia} \\[.3cm]
{\normalsize (a) http://www.eecs70.org} \\[.15cm]
(b) Homework: 10\%\\
\indent \ Midterm 1: 25\%\\
\indent \ Midterm 2: 25\%\\
\indent \ Final: 40\%\\[.25cm]

{\Large 2 Course Policies} \\[.3cm]
(a) Yes. As said in Course Policies, ``you must always write up the solutions on your own,'' but Alice and Bob wrote up a solution together. \\[.15cm]
(b) No. As long as Dan doesn't write his code on the board, Carol is a perfect example for getting help from peers, as she not only writes up her solutions alone in her own words, but also credits the person who has inspired her. \\[.15cm]
(c) No. As long as Erin doesn't copy the website verbatim, she is a perfect example for utilizing online resources since she doesn't copy the answer directly. Instead, she first understands the solution and then writes her solution alone, crediting the resource. \\[.15cm]
(d) Yes. Both Frank and Grace violated the policy. For Frank, he copied directly from Grace's work, which is not allowed as Course Policies said, ``At no time should you be in possession of another student's solution.''; for Grace, she shouldn't have given her written solutions. She could explain her ideas and approach, but not show her code. \\[.15cm]
(e) Yes. Similar to (d), both violated the policy. As said in Course Policies, ``At no time should you be in possession of another student's solution.'' Irene shouldn't use Heidi's solution to write her own, and Heidi shouldn't have sent her work.\\[.25cm]

{\Large 3 Use of Piazza} \\[.3cm]
(a) 13 \\[.15cm]
(b) Hi xxx (student's name), you should probably go to the office hours or the homework party this week. This question can't be explained within 5 minutes.\\[.25cm]

{\Large 4 \LaTeX} \\[.3cm]
(a) $ \forall x \exists y \big((P(x) \land Q(x, y)) \Longrightarrow x \leq \sqrt{y} \big) $ \\[.15cm]
(b) $ \sum\limits_{i=0}^{k} i = \frac{k(k+1)}{2} $

\end{document}
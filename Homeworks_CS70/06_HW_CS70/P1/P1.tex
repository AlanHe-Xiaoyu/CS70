\documentclass{article}
\usepackage[left=3cm, right=3cm, top=3cm]{geometry}
\usepackage{amssymb}
\usepackage{amsmath}
\usepackage{xcolor}
\begin{document}

I worked alone without getting any help, except asking questions on Piazza and reading the Notes of this course. \\[1cm]
{\Large 1 Polynomial Practice} \\[.5cm]
(a) \\[.1cm]
(i) At least: \textbf{0}. At most: \textbf{max(deg($f$), deg($g$))} \\[.1cm]
\indent Since we can't guarantee that the resulting polynomial $f + g$ always has a real root, so the least number of roots is 0. \\[.1cm]
\indent WLOG, let deg($f) \leq $ deg($g$). Consider $f + g$, since it is non-zero, so it is a polynomial whose degree is at most deg($g$), and since a polynomial with degree $d$ has at most $d$ roots, so $f + g$ has at most deg($g$) roots. Vice versa for the case of deg($f) >$ deg($g$), which gives us that the polynomial $f + g$ has at most max(deg($f$), deg($g$)) roots. \\[.3cm]
(ii) At least: \textbf{0}. At most: \textbf{deg($f$) + deg($g$)} \\[.1cm]
\indent Consider $f = g = x^2 + 1$, so $f, g$ have no roots, so $f\cdot g$ also have no roots, which implies that $f\cdot g$ is not guaranteed to have any roots, so the least number of roots is 0. \\[.1cm]
\indent Let $a =$ deg$(f)$ and $b =$ deg$(g)$. Since $x^a\cdot x^b$ = $x^{a+b}$, so the degree of the polynomial $f\cdot g$ is $a + b$, which is deg($f$) + deg($g$). Again, using the property of polynomials, so $f\cdot g$ has at most deg($f$) + deg($g$) roots. \\[.3cm]
(iii) At least: \textbf{0}. At most: \textbf{deg($f$) $-$ deg($g$)} \\[.1cm]
\indent Again, we can't guarantee that $f/g$ always has a root. For example, let $f = x^2 + 1$ and $g = 1$, so $f/g = x^2 + 1$ does not have a root. So, the least number of roots is 0. \\[.1cm]
\indent Let $a =$ deg$(f)$ and $b =$ deg$(g)$. Since $\frac{x^a}{x^b}$ = $x^{a-b}$ and that we are given that $f/g$ is a polynomial, so the degree of the polynomial $f/g$ is $a - b$, which is deg($f$) $-$ deg($g$). Again, using the property of polynomials, so $f/g$ has at most deg($f$) $-$ deg($g$) roots. \\[.5cm]
(b) \\[.1cm]
(i) No, it isn't. \\[.1cm]
\indent We proceed by providing a counterexample. Consider $p = 2,\ f(x) = x,\ g(x) = x + 1$. For all $x\in GF(p)$, namely, $x_1 = 0$ or $x_2 = 1$, we have that when $x_1 = 0$, $f\cdot g(x) = 0\cdot1 = 0$; when $x_2 = 1$, $f\cdot g(x) = 1\cdot2\equiv0\pmod{2}$, which gives us that $f\cdot g = 0$. \\[.1cm]
\indent Yet, when $x = 1$, $f(x) = 1\neq0$; when $x = 0$, $g(x) =1\neq0$, which implies that $f\neq0$ and $g\neq0$. Thus, this is a counterexample, so if $f,g$ are polynomials over $GF(p)$ and $f\cdot g = 0$, it isn't necessarily true that either $f = 0$ or $g = 0.$ \\[.3cm]
(ii) Direct Proof. \\[.1cm]
\indent $\textit{Proof.}$ We proceed by a direct proof. Suppose $f$ be a polynomial over $GF(p)$, and that deg$(f)\geq p$. \\[.1cm]
\indent Now, let $f(i) = y_i\ \forall i\in\{0,1,...,p-1\}$, which means that we have $p$ pairs $(0, y_0), (1, y_1), ..., (p-1, y_{p-1}$ with all the $x_i$ distinct. Thus, by Property 2 of polynomials, so there is a unique polynomial $h(x)$ of degree (at most) $p - 1$, which means that $h$ is a polynomial with deg$(h)\leq p-1 < p$ such that $f(x) = h(x)$ for all $x\in\{0,1,...,p-1\}$, as desired. Thus, such a polynomial $h$ exists. \\[.1cm]
Q.E.D. \\[.3cm]
(iii) $p^d$ \\[.1cm]
\indent For a polynomial $f$ with degree $d < p$, so $f(x) = a_dx^d + a_{d-1}x^{d-1} + \cdot\cdot\cdot + a_1x + a_0,$ where $a_i\in\{0,1,...,p-1\}\ \forall\ i\in\mathbb{N}, i\leq d$. \\[.1cm]
\indent Then, the only constraint on $f$ is that $f(0) = a_0 = a$ where $a\in\{0,1,...,p-1\}$. So only one coefficient, $a_0$, of $f$ is set, which means that $\forall\ 1\leq i\leq d, i\in\mathbb{N}$, we have that $a_i$ could be any value in the set $\{0,1,...,p-1\}$. Thus, there are $p$ possible values for each coefficient $a_i$ of $f$, and there are $d$-many coefficients we could assign values arbitrarily (as long as they are in $\{0,1,...,p-1\}$), which implies that there are $p^d$-many such polynomials $f$. \\[.5cm]
(c) $f = 4x^2 + 1$. There are $\textbf{25}$ such polynomial.\\[.3cm]
\textit{Part (1) Finding one such polynomial $f$} \\[.1cm]
\indent We first use Lagrange interpolation to find one such $f$. So, we have that $8^{-1}\equiv2\pmod{5}$ with $8\cdot2 = 16 = 5\cdot3 + 1$, and similarly, $(-4)^{-1}\equiv1\pmod{5}$ with $-4\cdot1 = -4 = 5\cdot(-1) + 1$. \\[.1cm]
\indent Thus, using Lagrange interpolation, we have: \\[.1cm]
\indent \indent $\Delta_1(x) = \frac{(x-2)(x-4)}{(0-2)(0-4)} = 
\frac{x^2-6x+8}{8} = 2(x^2-6x+8) = 2x^2 - 12x + 16 \\[.1cm]
\indent \indent \Delta_2(x) = \frac{(x-0)(x-4)}{(2-0)(2-4)} = 
\frac{x^2-4x}{-4} = 1(x^2-4x) = x^2 - 4x \\[.1cm]
\indent \indent \Delta_3(x) = \frac{(x-0)(x-2)}{(4-0)(4-2)} = 
\frac{x^2-2x}{8} = 2(x^2-2x) = 2x^2 - 4x \\[.1cm]$
\indent Thus, the polynomial $f(x)$ is therefore given by: \\[.1cm]
\indent $f(x) = 1\cdot\Delta_1(x) + 2\cdot\Delta_2(x) + 0\cdot\Delta_3(x) = 
1(2x^2 - 12x + 16) + 2(x^2 - 4x) + 0(2x^2 - 4x) = (2+2)x^2 + (-12-8)x + 16 = 4x^2 - 20x + 16 \\[.1cm]$
\indent Since $f$ is a polynomial over $GF(5)$, and that we have $-20 = 5\cdot(-4) + 0, 16 = 5\cdot3 + 1$, which gives us that $-20\equiv0\pmod{5}, 16\equiv1\pmod{5}$, so we have that $f(x) = 4x^2 + 1$. \\[.3cm]
\textit{Part (2) Showing that there are 25 such polynomials} \\[.1cm]
\indent Since we define $f$ over $GF(5)$, so the value of $x$ is also confined within $\{0, 1, 2, 3, 4\}$. Now, since $f(0) = 1, f(2) = 2, f(4) = 0,$ so we only have $x = 1, x = 3$ that we can arbitrarily assign values to, with $f(x)$ still confined within $\{0, 1, 2, 3, 4\}$. So, we have $5\cdot5 = 25$ different combinations of value assignments to $f(x), 0\leq x\leq4$. Then, by the property of polynomials, we can construct a unique polynomial based on each of these assignments, which means that there are 25 different such polynomials.


\end{document}
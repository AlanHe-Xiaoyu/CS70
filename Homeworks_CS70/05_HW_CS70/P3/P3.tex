\documentclass{article}
\usepackage[left=3cm, right=3cm, top=3cm]{geometry}
\usepackage{amssymb}
\begin{document}

{\Large 3 Squared RSA} \\[.5cm]
(a) Direct Proof \\[.2cm]
\textit{Proof.} We proceed by a direct proof. Given that $p$ is prime and $a,p$ is coprime, let $a = kp + r$ where $k\in\mathbb{Z}, 1<r<p$ since $r\neq0$, so $r\in\{1,2,...,p-1\}$ and that $a\equiv r\pmod{p}$. Thus, using Fermat's Little Theorem, we have that $a^{p-1}\equiv r^{p-1}\equiv1\pmod{p}$, and so $a^p\equiv1\cdot p\equiv p\pmod{p}$. \\[.1cm]
\indent Let $A$ denote the set of non-zero integers mod $p^2$, i.e. $A = \{0, 1, 2, ..., p^2 - 1\}$. Let $S$ be the subset of $A$ where we exclude the multiples of $p$, i.e. $0, p, 2p, 3p, ..., (p-1)p$. There are $p-1$ such multiples of $p$, so $S$ has $p^2 - p = p(p-1)$ elements, and contains all the elements from 0 to $p^2-1$ that are not multiples of $p$, i.e. $S = \big\{kp+r\mid k\in\{0,1,2,...,p-1\}, r\in\{1,2,...,p-1\}\big\}$. \\[.1cm]
\indent Consider the sequence of numbers as we multiply each element of $S$ by $a$, given that $a,p$ are coprime, which represents $S^* = \big\{a(kp+r)\mid k\in\{0,1,2,...,p-1\}, r\in\{1,2,...,p-1\}\big\}$. We claim that these are all distinct modulo $p^2$. \\[.1cm]
\indent First, we provide the proof for a smaller claim, that none of these numbers could be a multiple of $p$. Since $\forall k,r\leq p-1, k\in\mathbb{N}, r\in\mathbb{Z^+},$ we have that $a(kp+r) = akp + ar\equiv ar\pmod{p}.$ Since $p$ is prime and that gcd$(p,a) = 1,$ gcd$(p,r)$ = 1, so we have gcd$(p, ar) = 1,$ which gives us that $a(kp+r)\equiv ar\not\equiv0\pmod{p}.$ Therefore, for any $e\in S^*,$ then $p\nmid e.$ Moreover, let $e\equiv x\pmod{p^2}$, so $e = qp^2 + x, q\in\mathbb{Z}$. If $p\mid x$, then with the fact that $qp^2 = (qp)p, qp\in\mathbb{Z}$, so $p\mid(qp^2),$ and so $p\mid(qp^2 + x) = e$, which gives the contradiction. Thus, $p\nmid x$. \\[.1cm]
\indent Thus, $\forall e\in S^*,$ let $e\equiv x\pmod{p^2}$, then $p\nmid x$. So, there are $p^2 - p$ possible remainders modulo $p^2$ (just as how I proved the number of elements in $S$). Now, we claim that for all elements in $S^*$, they are distinct modulo $p^2$. \\[.1cm]
\indent Assume for a contradiction that $\exists e_1, e_2\in S^*, e_1\neq e_2$, and $e_1\equiv e_2\pmod{p^2}$. 
So, $p^2\mid(e_1-e_2).$
Let $e_1 = a(k_1p+r_1), e_2 = a(k_2p+r_2)$ where $k_1, k_2,r_1, r_2\leq p-1, k_1, k_2\in\mathbb{N}, r_1, r_2\in\mathbb{Z^+}$, and let $e_1-e_2 = k_pp^2, k_p\in\mathbb{Z}.$
So $e_1 - e_2 = a(k_1p+r_1) - a(k_2p+r_2) = a(k_1-k_2)p + a(r_1-r_2)$
Since $p^2\mid(e_1-e_2),$ so $p\mid(e_1-e_2),$ and since $a,k_1,k_2\in\mathbb{Z},$ so $a(k_1-k_2)\in\mathbb{Z}$, so $p\mid a(k_1-k_2)p,$ so $p\mid a(r_1-r_2)$, since gcd$(p,a)=1$, so $p\nmid a,$ and with $p$ being prime, so $p\mid(r_1-r_2)$. If $r_1\neq r_2$, then $0<|r_1-r_2|<p-2$, which means that $p\nmid(r_1-r_2)$, implying contradiction, so $r_1 = r_2.$ \\[.1cm]
\indent Then, since $e_1\neq e_2,$ so $k_1\neq k_2.$ With a similar argument as above, so $p\nmid(k_1-k_2)$, and let $R$ be this assertion. However, with $r_1 = r_2,$ so $a(r_1-r_2) = 0$, so $k_pp^2 = e_1-e_2 = a(k_1-k_2)p.$ Since $p\neq0$, divide both sides by $p$ and we have $k_pp = a(k_1-k_2),$ so $p\mid a(k_1-k_2)$. Again, since gcd$(p,a) = 1$, so $p\nmid a$, and since $p$ is prime, so $p\mid(k_1-k_2)$, which implies $\neg R$. So, $R\land\neg R$ holds, reaching a contradiction, so for all elements in $S^*,$ they are distinct modulo $p^2.$ \\[.1cm]
\indent Thus, the set of numbers $S' = S^*$ mod $p^2 = \big\{a(kp+r)$ mod $p^2\mid k\in\{0,1,2,...,p-1\}, r\in\{1,2,...,p-1\}\big\}$ includes every element of $S$ exactly once, so it shuold be exactly the same as $S$, with possibly different order. \\[.1cm]
\indent Now, first take the product of all elements of $S$, mod $p^2$, would give us:
\begin{center}
$1\cdot2\cdot\cdot\cdot(p-1)\cdot(p+1)\cdot\cdot\cdot(2p-1)\cdot(2p+1)\cdot\cdot\cdot\cdot\cdot\cdot(p^2-1) =
\prod\limits_{e\in S} e\pmod{p^2}$.
\end{center}
On the other hand, take the product of all elements of $S'$, mod $p^2$, would give us:
\begin{center}
$a\cdot2a\cdot\cdot\cdot(p-1)a\cdot(p+1)a\cdot\cdot\cdot(2p-1)a\cdot(2p+1)a\cdot\cdot\cdot\cdot\cdot\cdot(p^2-1)a = \prod\limits_{e\in S} ea = a^{|S|}\cdot\prod\limits_{e\in S} e $ \\[.1cm]
$ = a^{p(p-1)}\cdot\prod\limits_{e\in S} e \pmod{p^2}.$
\end{center}
Thus, we have:
\begin{center}
$\prod\limits_{e\in S} e\equiv a^{p(p-1)}\cdot\prod\limits_{e\in S} e \pmod{p^2}$.
\end{center}
\indent Then, since every element of $S$ is coprime with $p^2$, so they would each have an inverse mod $p^2$, and thus, $\prod\limits_{e\in S} e$ would have an inverse mod $p^2.$ Therefore, multiplying both sides of the above equation by the inverse of $\prod\limits_{e\in S} e\pmod{p^2}$, we have that $a^{p(p-1)}\equiv1\pmod{p^2},$ as desired. \\[.1cm]
Q.E.D. \\[.5cm]
(b) Direct Proof \\[.2cm]
\textit{Proof.} We proceed by a direct proof. \\[.1cm]
\indent Consider the new RSA scheme where the public key is $(N = p^2q^2, e)$ with $e$ being relatively prime to $p(p-1)q(q-1)$, and the private key being $d = e^{-1}\pmod{p(p-1)q(q-1)}$. Also, we have our message $x$ being relatively prime to both $p$ and $q$, i.e. $x^{ed}\equiv x\pmod{N}.$ To prove that the scheme is correct, We have to show that $D(E(x))\equiv x\pmod{N}$ for every possible message $x\in\{0,1,...,N-1\}.$\\[.1cm]
\indent By definition of RSA, since we didn't change the definition of encyprtion/decryption functions, so the encrypted message $y = E(x)\equiv x^e\pmod{N}$, so $D(y) = D(E(x))\equiv(x^e)^d\equiv x^{ed}\pmod{N}.$ Then, since we are given that $x$ is relatively prime to both $p$ and $q$, i.e. $x^{ed}\equiv x\pmod{N},$ so $D(E(x))\equiv x^{ed}\equiv x\pmod{N}$, as desired. \\[.1cm]
\indent Thus, the new scheme is correct $\forall x$ relatively prime to both $p$ and $q$. \\[.1cm]
Q.E.D. \\[.5cm]
(c) Direct Proof \\[.2cm]
\textit{Proof.} We proceed by a direct proof. Suppose that we can break the new squared RSA scheme, i.e. if given $p^2q^2$, then we can deduce $p(p-1)q(q-1).$ \\[.1cm]
\indent Then, if we're given $pq,$ by squaring it, we can calculate $(pq)^2 = p^2q^2.$ Now, we know that given $p^2q^2,$ we can deduce $p(p-1)q(q-1)$. Since $p(p-1)q(q-1) = (pq)(p-1)(q-1),$ so dividing $p(p-1)q(q-1)$ by $pq$ would give us $(p-1)(q-1).$ Since the information $pq$ is given to us, so we can deduce $(p-1)(q-1)$ in this situation. \\[.1cm]
\indent Thus, if the new scheme, squared RSA, can be broken (i.e. if given $p^2q^2$, then we can deduce $p(p-1)q(q-1)$), then if we're given $pq$, we can also deduce $(p-1)(q-1)$, which implies that the normal RSA would also be broken, as desired. \\[.1cm]
Q.E.D.

\end{document}